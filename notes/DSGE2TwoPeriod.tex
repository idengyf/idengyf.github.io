
\documentclass[]{article}
\usepackage{graphicx}
%opening
\title{A Two-Period Model of Money Non-neutrality}
\author{Tao Wang}

\begin{document}

\maketitle

\section{The intuition}

The key prediction in the NK model is that money is not neutral, i.e. monetary policy shock, say an unexpected interest rate hike has real effects on the economy. The output drops, the unemployment rate goes up, inflation goes down, etc.  Why is it the case?
First and foremost, it is important to know the interest rate effect in the canonical NK model comes from inter-temporary substitution. Consumer’s optimal decision of consumption growth is always governed by an Euler equation. A higher real rate represents a higher price of consumption today versus tomorrow, hence inducing less consumption today relative to tomorrow. In the equilibrium absence of any disturbances, the real rate between today and tomorrow should adjust such that the consumption is equal to production both today and tomorrow. This is solely determined by the real block of the economy. With nominal rigidity, however, the prevailing real rate in the market, which is the nominal rate minus expected inflation, is not necessarily equal to that market-clearing rate. This leads to the real effects of the nominal interest rate. 

It is always easy to think that the nominal rate is set exogenously by a policymaker. Now imagine an  exogenous nominal interet hike. What impacts will it have? The ambiguity here might come from the effect of the nominal rate on the real rate.  

To think this through, it is the easiest to ask what the economy would go following such an interest rate hike if the price is flexible. Despite the nominal rate change, the total recourses generated by the economy do not change, and both the demand and supply curves do not shift. Therefore, the equilibrium real interest rate does not change. Now a higher nominal rate without price adjustment may dampen the demand, hence inducing a steeper consumption growth: less today versus tomorrow. Given the same output schedule in two periods, the natural force that helps clear the economy is to allow the price to drop today versus tomorrow. This is exactly equivalent to expected inflation going up. Flexible price adjustment implies this happens immediately therefore, no change in real interest rate. 

Now think about the other extreme with full rigidity. Under full rigidity, the expected inflation does not change. Therefore, the real rate fully reflects the increase in the nominal rate. Then according to the Euler equation, a higher nominal rate requires a higher consumption growth, thus a lower consumption today versus tomorrow. The market is not clearing today in that the demand is lower than the supply. This is contractionary to the economy. This helps illustrate why nominal rigidity is such a central assumption that makes monetary policy have real effects. Nominal rate can influence the real rate in the short run because the price does not adjust instantly. 

Between the two extreme cases discussed above, the NK model allows the price to partially adjust each period. This essentially expands the two periods to multiple periods and adds additional dynamics to the model. 


Regardless of the specific form it takes, what is needed essentially across models is simply price rigidity. For instance, the working model of NKDSGE adopts on the modeling device of Calvo pricing. A forward-looking firm sets a price by taking into account future inflation and change in its own marginal cost. The firm-specific marginal cost change is proportional to the change in the marginal cost of the aggregate economy. Also, it is shown that the output gap is proportional to the change in marginal cost. This is how we get the NKPC curve. Current inflation is determined by expected inflation and the output gap. 

There is a presence of expected inflation in both household and firm’s problems. The question is how are they determined? In order to pin down the system, mathematically we need to solve forward the inflation expectation. In practice, that means we need an exogenous policy rule that tells how the market forms the expectation. This is why we need to hit inflation hard enough to arrive at a stable solution to the system. 

\section{Graphical illustration}

Figure \ref{period2} shows the intuition discussed above. Plot the supply and demand today and tomorrow, separately. We use consumption tomorrow as the numeraire to denote the price of consumption today, which is exactly the real interest rate. This is convenient because all changes happen today. Therefore, the market-clearing ``price'' is $R^*$ today and 1  for tomorrow. 

Now imagine that the nominal interest rate unexpectedly increases. Neither demand and supply curves shift. Therefore, the real interest rate that clears the market is still $R^*$.  Under full price flexibility, the change in nominal rate be counterbalanced by the change in price/expected inflation and the resulting real interest rate $R~$ is still equal to $R^*$.  In contrast, with full rigidity, nominal rate hike results in the same increase in real rate $R-$ since price today cannot change.  Therefore, this leads to lower demand than supply. The output gap is the difference between the current demand and the equilibrium output.

\begin{figure}[ht]
	\centering
	\caption{Following an (unexpected) nominal rate hike with/without nominal rigidity}
		\label{period2}
	\includegraphics[width=\textwidth]{TwoPeriodNK.JPG}
\end{figure}
\end{document}
